\documentclass{article}
\usepackage{amsmath}
\usepackage{amssymb}
\usepackage{amsthm}
\usepackage{enumerate}
\usepackage{color}
\usepackage{hyperref}
\hypersetup{
    colorlinks,
    citecolor=black,
    filecolor=black,
    linkcolor=black,
    urlcolor=black
}

\title{Math 105a: Real Analysis}
\author{Edwin Lin}
\date{Lecture notes from Math 105a}

\theoremstyle{theorem}
\theoremstyle{definition}
\theoremstyle{proposition}
\theoremstyle{remark} 

\newtheorem{theorem}{Theorem}[section]
\newtheorem{definition}{Definition}[section]
\newtheorem{proposition}{Proposition}[section]
\newtheorem*{remark}{Remark}

\begin{document}
\maketitle
\newpage
\tableofcontents
\newpage

\section{Lecture 1}
\subsection{Totally Ordered set}


\begin{definition}[Totally Ordered Sets]
A totally ordered set  is a set $S$ together with a relation $<$ such that
\begin{enumerate}[I.]
    \item (Trichotomy) For all $x,y \in S$, exactly one of the following holds 
    \begin{enumerate}[(i)]
        \item $x < y$
        \item $x = y$ 
        \item $x > y$
    \end{enumerate}
    \item (Transitivity) if $x,y,z \in S$ and if $x < y$ and $y < z$, then $x < z$.
\end{enumerate}
We write $x \leq y$ if $x < y$ or $x = y$.

\end{definition}
\begin{remark}
    We denote totally ordered set as ordered set.
\end{remark}

\begin{definition}[Bounds]
Let $E \subset S$, where $S$ is an ordered set.
\begin{enumerate}[(i)]
    \item If there exists some $b \in S$ such that $x \leq b$ for all $x \in E$, then $b$ is an \textbf{upper bound} for the set $E$. We say $E$ is \textbf{bounded above}.
    \item If there exists some $b \in S$ such that $b \leq x$ for all $x \in E$, then $b$ is a \textbf{lower bound} for the set $E$. We say $E$ is \textbf{bounded below}.
    \item If there exists an upper bound for $E$, call it $b_{0}$ such that $b_{0} \leq b$ for all other upper bounds $b$ for $E$, then $b_{0}$ is the \textbf{least upper bound} for $E$, or the \textbf{supremum}.
    \item If there exists $b_{0} \in S$ such that $b \leq b_{0}$ for all $b \in E$, then $b_{0}$ is the \textbf{greatest lower bound} \textbf{infimum} for $E$.
    \item If $E$ has an upper bound and lower bound then it is \textbf{bounded}.
\end{enumerate}
\end{definition}

\begin{proposition}
    Let E be a subset of the ordered set $E$. If $\inf E$ and $\sup E$ exists, then they are unique.
\end{proposition}

\begin{proof}
    $ $\newline
    Suppose $b_0$ is an infimum for $E$ and $b_0^\prime$ is an infimum for $E$. We want to show $b_0 = b_0^\prime$. \\ \\
    Since $b_0$ is an infimum for $E$, then by definition 1.2 $b_0 \geq b$ for all other lower bounds $b$ for $E$. $b_0^\prime$ is also an infimum for $E$, meaning $b_0^\prime$ is a lower bound for $E$. Thus $b_0 \geq b_0^\prime$. Similarly, $b_0^\prime$ is greater than all other lower bounds, so $b_0^\prime \geq b_0$. Thus $b_0 = b_0^\prime$ and thus is unique.
\end{proof}

\begin{definition}[Least Upper Bound Property]
   An ordered set $S$  has the \emph{least upper bound property} if every non-empty subset $E \subset S$ that is bounded above has a least upper bound.
\end{definition}

\begin{remark}
    Check out completeness property
\end{remark}

\begin{definition}[Fields]
   A set $F$  is a field if it has two operations defined on it, $+$ and $\cdot$, and it satisfies the following axioms.
   \begin{enumerate}[(i)]
       \item If $x \in F$ and $y \in F$, then $x + y \in F$
       \item $x + y = y + x$ for all $x,y \in F$
       \item $(x + y) + z = x + (y + z)$ for all $x,y,z \in F$
       \item There exists an additive identity element, $0$, such that $x + 0 = x$ for all $x \in F$
       \item There exists an additive inverse, we denote $-x \in F$, such that $x + (-x) = 0$ for all $x \in F$
       \item If $x,y \in F$, then $xy \in F$
       \item $xy = yx $ for all $x,y \in F$
       \item $(xy)z = x(yz)$ for all $x,y,z \in F$
       \item There exists an multiplicative identity element, $1$, such that $x \cdot 1 = x$ for all $x \in F$
       \item For all non-zero $x\in F$, there exists a multiplicative inverse $x^{-1} \in F$ such that $x\cdot x^{-1} = 1$
       \item $1 \neq 0$
   \end{enumerate}
\end{definition}
\begin{theorem}
   There exists a unique ordered field containing $\mathbb{Q}$  that satisfies the least upper bound property. We call this field $\mathbb{R}$.
\end{theorem}
\begin{proposition}
    If $x \in \mathbb{R}$ such that $x \leq \epsilon$ for all $\epsilon \in \mathbb{R}$ where $\epsilon > 0$, then $x \leq 0$.
\end{proposition}
\end{document}